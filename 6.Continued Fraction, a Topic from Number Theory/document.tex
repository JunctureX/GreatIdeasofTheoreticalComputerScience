\documentclass[12pt, a4paper, oneside]{ctexart}
\usepackage{amsmath, amsthm, amssymb, bm, color, framed, graphicx, hyperref, mathrsfs}
\usepackage{listings}
\usepackage{ctex}

\usepackage{graphicx}
\usepackage{float}
\usepackage{subfigure}

\usepackage{palatino}
\usepackage{tikz}
\usetikzlibrary{shapes.geometric, arrows, graphs, positioning, quotes}

\usepackage{xcolor}
\usepackage{xcolor}
\usepackage{color}
\definecolor{dkgreen}{rgb}{0,0.6,0}
\definecolor{gray}{rgb}{0.5,0.5,0.5}
\definecolor{mauve}{rgb}{0.58,0,0.82}

\usepackage{multirow}
\usepackage{booktabs}

\title{\textbf{TCS作业六:Continued Fractions, a Topic from Number Theory}}
\author{计试2101仲星焱}
\date{\today}

\begin{document}
	\textbf{TCS作业六:Continued Fraction, a Topic from Number Theory}
	
	计试2101仲星焱
	
	\today
	
	\paragraph{6.1}
	$$
	\begin{aligned}
		\frac{p_0}{q_0}&=[a_0]\\
		\frac{p_1}{q_1}&=a_0+1/a_1=[a_0,a_1]
	\end{aligned}
	$$
	
	考虑数学归纳,假设所有 $n<k,\frac{p_n}{q_n}=[a_0,\dots,a_n]$ 成立,对于 $n=k$ $$
	\begin{aligned}
		\frac{p_k}{q_k}&=\frac{a_{k}(a_{k-1}p_{k-2}+p_{k-3})+p_{k-2}}{a_{k}(a_{k-1}q_{k-2}+q_{k-3})+q_{k-2}}\\&=\frac{(a_{k-1}+1/a_k)p_{k-2}+p_{k-3}}{(a_{k-1}+1/a_k)q_{k-2}+q_{k-3}}\\&=[a_0,\dots,a_{k-1}+\frac{1}{a_k}]\\&=[a_0,\dots,a_k]
	\end{aligned}
	$$
	
	\paragraph{6.2}
	考虑数学归纳,假设对于 $n<k$,在 $a_n<a_n'$ 时,对于奇数 $n$ 有 $[a_0,a_1,\dots,a_n]>[a_0,a_1,\dots,a_n']$,对于偶数 $n$ 相反。
	
	在 $a_n<a_n'$ 时,有 $a_{n-1}+\frac{1}{a_n}>a_{n-1}+\frac{1}{a_n'}$,令 $n=k$。
	
	若 $k$ 为偶数则 $k-1$ 为奇数,有 $[a_0,a_1,\dots, a_{n-1}+1/a_n]<[a_0,a_1,\dots, a_{n-1}+1/a_n']$ ,进而$[a_0,a_1,\dots,a_n]<[a_0,a_1,\dots,a_n']$
	
	 $k$ 为奇数的时候同理。
	
	\paragraph{6.3}
	
	对于任意正有理数 $$
	\begin{aligned}
		\frac{p}q & =[a_0,a_1,\dots,a_N]\\
		&=[a_0,a_1,\dots, a_N+1,-1]
	\end{aligned}$$
	
	命题成立,证毕。
	
	\paragraph{6.4}
	
	先考虑 $k=0,l>0$ 的情况,需要找到 $t=[a_0,a_1,\dots,a_{l-1},a_0,\dots]$ 的解。由连分数的性质,我们得到$t=[a_0,a_1,\dots,a_{l-1},t]$,接下来我们考虑化简得到的方程形式。
	
	通过观察不难发现,化简中途和结果可以表示为 $$\frac{p_it+u_i}{q_it+v_i}=[a_i,\cdots,a_{l-1},t]$$ 其中 $p_0=v_0=1,u_0=q_0=0$,考虑使用数学归纳证明这个关系,则不难发现只需要令 $$\begin{aligned}p_{i+1}&=q_i\\v_{i+1}&=u_i\\q_{i+1}&=p_i-a_iq_i\\v_{i+1}&=u_i-a_iv_i\end{aligned}$$ 即可,如此最终可以得到 $$\frac{p_lt+u_l}{q_lt+v_l}=t$$
	为一个不超过二次的方程,由此我们知道 $\exists \hat{u},\hat{v},\hat{w},\hat{x}\in \mathbb Z,\textit{s.t.} t=\frac{\hat{v}+\hat{w}\sqrt{\hat{x}}}{\hat{u}}$。
	
	现在考虑 $k>0$ 的情况,设循环开始截取的连分数极限值为 $t_k=[a_k,a_{k+1},\dots,a_{k+l-1},a_k,\dots]$,设最终求值为 $r=[a_0,a_1,\dots]$
	
	$$
	\begin{aligned}
		r&=[a_0,a_1,\dots,a_{k-1},t_k]\\
		&=[a_0,a_1,\dots,a_{k-2},t_{k-1}]\\
		&=\cdots\\
		&=t_0
	\end{aligned}
	$$
	
	其中 $t_i=a_i+\frac{1}{t_{i+1}}$,只需要证明所有 $t_i=\frac{v_i+w_i\sqrt{x}}{u_i}$ 可以递推即可,直接递推不难发现
	$$\begin{aligned}
		u_{i-1}&=v_i^2-w_i^2x\\
		v_{i-1}&=u_iv_i+a_iu_{i-1}\\
		v_{i-1}&=-u_iw_i
	\end{aligned}$$
	
	故最终可以表示为 $\frac{v+w\sqrt x}u$,
	证毕。
	
	\paragraph{6.5}
	
	引理1:{\kaishu 整系数多项式有可数无穷多}
	
	设从小到大第 $i$  个质数为 $p_i$。由唯一分解定理,任意正整数 $n=\prod\limits_{i=1}^\infty p_i^{a_i}$,其中 $a_i$ 对于任意 $n$ 有唯一确定的值。
	
	设整系数多项式的集合为 $\mathbb F$ 考虑如下映射:$f:\mathbb N_+\rightarrow\mathbb F$
	
	$$
	f(n)=f(\prod_{i=1}^\infty p_i^{a_i})=\sum_{i=0}^\infty a_{i+1}x^i
	$$
	
	显然,这是一个双射。由此,我们证明 $\mathbb F$ 与 $\mathbb N_+$ 等势。即整系数多项式有可数无穷多。
	
	引理2:{\kaishu 代数数有可数无穷多}
	
	设 $R_p$ 表示多项式 $p$ 的所有根的集合,设所有代数数的集合为 $A$,有 $$A=\bigcup\limits_{p\in\mathbb F}R_p$$ 即可数个有限集合的并,故 $|A|=\aleph_0$
	
	引理3:{\kaishu 超越数有不可数无穷多}
	
	设 $T$ 表示超越数集合, $T = \mathbb R / A$,而 $|\mathbb R|=\aleph,|A|=\aleph_0$,故 $|T|=\aleph$ 为不可数无穷多。
	
	故不存在单射 $f:T\rightarrow A$。
	
	\paragraph{6.6}
	
	考虑反证法,假设不满足条件的 $n$ 存在,则 $\exists k\in mathbb Z_+ k<2^{n-1}\sqrt 3 < k + \frac{1}{2^{n+2}}$,平方得到: $$\begin{aligned}
		k^2<2^{2n-2}\cdot 3&<k^2+\frac{k}{2^{n+1}}+\frac{1}{2n+4}\\
		&<k^2+\frac{\sqrt{3}}4+\frac{1}{16}\\
		&<k^2+1
		\end{aligned}
		$$
	
	而 $k^2$ 与 $k^2+1$ 之间不可能有整数 $2^{2n-2}\cdot 3$,故不存在不满足条件的 $n$,原命题成立。
	
	\paragraph{6.7}
	
	考虑证明其逆否命题:{\kaishu 如果$\xi$ 为有理数,则只存在有限组 $(a,b)$ 满足 $0<|\xi - \frac{a}b|<\frac{1}{b^\alpha}$。}
	
	一个比较自然想法是考虑 $b$,枚举 $a$。展开绝对值进行推导:
	$$
	\begin{aligned}
		-\frac{1}{b^\alpha}&<\xi-\frac{a}b<\frac{1}{b^\alpha}\\
		\Rightarrow b(\xi-\frac{1}{b^{\alpha}})&<a<b(\xi+\frac{1}{b^\alpha})
	\end{aligned}
	$$
	令 $\xi=\frac{p}q,\gcd(p,q)=1$,$$\frac{bp}q-\frac{1}{b^{\alpha-1}}<a<\frac{bp}q+\frac{1}{b^{\alpha-1}}$$
	对于足够大的 $b$,$$\frac{1}{b^{\alpha-1}}<\frac{1}q$$ 有
	$$
	\frac{bp-1}q<a<\frac{bp+1}q
	$$
	要使 $a$ 存在正整数取值,由于 $\gcd(p,q)=1$,故只可能为 $\frac{bp}q$,但是此时 $|\xi-\frac{a}b|=0$,不满足要求。
	
	故对于有理数 $\xi$ 来说,可能的 $b$ 取值存在上限,$b$ 只有有限个取值,单个 $b$ 也只有有限个可能的 $a$ 的取值。
	
	因此,若 $\xi$ 为有理数,可能的 $a,b$ 取值只有有限组。
	
	其逆否命题也成立,若 $a,b$ 取值有无限种可能,$\xi$ 为无理数。
	
	\paragraph{6.8}
	
	首先容易注意到所有偶数 $n,a_n>1$,所有奇数 $n,a_n\leq 1$。
	
	引理1:{\kaishu 所有 $a_n$ 互不相同}
	
	首先, $a_n=1$ 当且仅当 $n=1$,显然,从 $a_2>1,a_3<1$ 分奇偶进行数学归纳即可。
	
	然后,假设存在 $a_n=a_m$,我们找到一组 $n,m, s.t. n+m$ 最小。不难发现 $n,m$ 必同奇偶,且都大于 $1$,若同为偶数则 $a_{n/2}=a_{m/2}$,否则 $a_{n-1}=a_{m-1}$。与 $n+m$ 最小矛盾,即不存在 $a_n=a_m$。
	
	引理2:{\kaishu $\forall p,q \in \mathbb Z_+ \gcd(p,q)=1,\exists n\in \mathbb N_+, s.t. a_n=\frac{p}q$}。
	
	考虑利用构造法来找到这个 $n$,设 $\frac{p}q = [b_0,b_1,\dots,b_N]$,其中 $b_0$ 可以为 $0$,剩下的 $b_i$ 全部为正整数,显然,用欧几里得迭代得到的连分数表示唯一。
	
	则只需要令 $n=2^{b_0}(1+2^{b_1}(1+2^{b_2}(1+\cdots)))$,不难验证$a_n$ 的递推过程本质上就是连分数的计算过程, $a_n = \frac{p}q$。
	
	结合引理1和引理2,证毕。
	
\end{document}