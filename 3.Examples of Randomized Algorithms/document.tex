\documentclass[12pt, a4paper, oneside]{ctexart}
\usepackage{amsmath, amsthm, amssymb, bm, color, framed, graphicx, hyperref, mathrsfs}
\usepackage{listings}
\usepackage{ctex}

\usepackage{graphicx}
\usepackage{float}
\usepackage{subfigure}

\usepackage{palatino}
\usepackage{tikz}
\usetikzlibrary{shapes.geometric, arrows, graphs, positioning, quotes}

\usepackage{xcolor}
\usepackage{xcolor}
\usepackage{color}
\definecolor{dkgreen}{rgb}{0,0.6,0}
\definecolor{gray}{rgb}{0.5,0.5,0.5}
\definecolor{mauve}{rgb}{0.58,0,0.82}

\usepackage{multirow}
\usepackage{booktabs}

\title{\textbf{TCS作业三:Examples of Ramdomized Algorithms}}
\author{计试2101仲星焱}
\date{\today}

\begin{document}
	\textbf{TCS作业三:Examples of Randomized Algorithms}
	
	计试2101仲星焱
	
	\today
	
	\paragraph{3.1} 考虑对T进行二次差分
	
	$$\Delta T(n) = 1 - \frac{2}{n(n-1)}\sum_{k=0}^{n-2} T(k)+ \frac{2} n T(n-1)$$
	$$\Delta^2T(n) = \frac{4}{n(n-1)(n-2)}\sum_{k=0}^{n-3}T(k)+\frac{2}n(T(n-1)-T(n-2)) > 0$$
	
	显然为下凸函数。
	
	\paragraph{3.2}
	考虑任何一个最小割,它必然将图分为两部分,考虑点对 $\forall 2\leq i\leq n,(1,i)$,由鸽巢原理,至少有一个点对被某个最小割分到了两个不同的连通分量当中。
	
	由最大流最小割定理,任意 $S-T$ 最大流的流量等于 $S-T$ 最小割的权值,故我们枚举 $i=2,3,\dots,n$,计算 $1-i$ 的最大流即可,根据最大流最小割定理,我们计算出所有 $1-i$ 最大流中最小的即为所求最小割。由于 $FF$ 算法复杂度为 $O(FE)$,此处为简单图故流量不超过 $n-1$,由此得到 $FF$ 在该图上复杂度为 $O(n^3)$,执行 $n$ 次则为 $O(n^4)$。
	
	以上解答完毕,下面是补充说明。
	
	简单证明最大流-最小割定理:
	
	引理1:{\kaishu 任意 $S-T$ 流 $F$ 的流量不会超过任意 $S-T$ 割 $C$ 的权值。}
	
	考虑对于 $C$ ,逐一去掉其中的边,同时每删去一条边就对 $F$ 的失效流量进行退流。不难发现最终 $F$ 全部退流,否则 $C$ 不可能是一个割。考虑每次删去的边的权值都大于等于退掉的流量,可知引理 $1$ 成立。
	
	引理2:{\kaishu 对于任意最大流 $F$,我们可以构造一个割与其权值相同。}
	
	考虑从 $S$ 开始进行深度优先搜索,如果当前前进经过的边为 $F$ 中满流的边,不前进,将边删去并加入 $C$ 中。显然 $C$ 是一个 $S-T$ 割,否则 $F$ 仍然可以通过剩余的网络中 $S-T$ 的通路进行增广,不可能是最大流。
	
	结合引理1和引理2可知任意图 $S-T$最大流 等于 $S-T$ 最小割。
	
	实际上,全局最小割存在更加简洁的 Stoer-wagner 算法,基于一般图最小割的更强的结论——不会出现超过两个连通分量,复杂度为 $O(n^3)$,由于和本题无关此处不做赘述。
	
	\paragraph{3.3}
	设运行 $k$ 次,设失败概率为 $p$,有 $$\begin{aligned}
		p\leq(1-\frac{2}{n(n-1)})^k\leq \epsilon\\
		k\geq \log_\epsilon (1-\frac{2}{n(n-1)})
	\end{aligned}$$
	
	\paragraph{3.4}
	
	考虑猴子排序,每次对其进行随机打乱,不妨认为是 $Knuth$ 洗牌法,也即 C++ 中 std::random\_shuffle 的实现。
	
	容易发现,单次排序成功的概率不小于 $1/n!$,当有重复元素时概率更高。则进行 $k$ 次算法内排序成功概率为 $1-(1-1/n!)^k$。
	
	\paragraph{3.5}
	考虑 Karger 算法的单次执行,其停机时得到一个割,而对于任意一个最小割,得到该最小割的概率是 $P=Pr[\textit{edge in  min-cut never picked}]\geq \frac{2}{n(n-1)}$。
	
	不难发现得到任意两个不同的最小割是互斥事件,设有 $k$ 个不同的最小割,由概率的性质我们知道 $Pr[\textit{Karger's algorithm ends with a min-cut}]\leq kP\leq 1 \Rightarrow k \leq {n\choose 2}$,证毕。
	
\end{document}