\documentclass[12pt, a4paper, oneside]{ctexart}
\usepackage{amsmath, amsthm, amssymb, bm, color, framed, graphicx, hyperref, mathrsfs}
\usepackage{listings}
\usepackage{ctex}

\usepackage{graphicx}
\usepackage{float}
\usepackage{subfigure}

\usepackage{palatino}
\usepackage{tikz}
\usetikzlibrary{shapes.geometric, arrows, graphs, positioning, quotes}

\usepackage{xcolor}
\usepackage{xcolor}
\usepackage{color}
\definecolor{dkgreen}{rgb}{0,0.6,0}
\definecolor{gray}{rgb}{0.5,0.5,0.5}
\definecolor{mauve}{rgb}{0.58,0,0.82}

\usepackage{multirow}
\usepackage{booktabs}

\title{\textbf{TCS作业五:Nash Equilibrium, a Solution Concept from Game Theory}}
\author{计试2101仲星焱}
\date{\today}

\begin{document}
	\textbf{TCS作业五:Nash Equilibrium, a Solution Concept from Game Theory}
	
	计试2101仲星焱
	
	\today
	
	\paragraph{5.1} 
	
	考虑如下的双人策略游戏:
	
	\begin{table}[htbp]
		\centering
		\begin{tabular}{|c|c|c|}
			\toprule
			& 1     & 2 \\
			\midrule
			1     & [0,1] & [1,0] \\
			\midrule
			2     & [1,0] & [0,1] \\
			\bottomrule
		\end{tabular}%
		\label{tab:addlabel}%
	\end{table}%

	考虑玩家 $A$ 执行两行动的概率分别为 $a_1,a_2$,玩家 $B$ 执行两行动的概率分别为 $b_1,b_2$。则 $A$ 收益为 $a_1b_2+a_2b_1$,$B$ 收益为 $a_1b_1+a_2b_2$ 符合零和游戏的模型,故存在混合策略纳什均衡,解得 $a_1=a_2=b_1=b_2=1/2$, 此为唯一的纳什均衡,也是混合策略纳什均衡。
	
	\paragraph{5.2}
	
	考虑如下的双人策略游戏:
	
	\begin{table}[htbp]
		\centering
		\begin{tabular}{|c|c|c|}
			\toprule
			& 1     & 2 \\
			\midrule
			1     & [1,1] & [1,0] \\
			\midrule
			2     & [0,1] & [0,0] \\
			\bottomrule
		\end{tabular}%
		\label{tab:addlabel}%
	\end{table}%
	
	此游戏中唯一的纳什均衡为双方都选择执行 $1$ 行动,此时双方都能获得 $1$ 的收益而任何一方选择 $2$ 行动只会白白送掉自己的收益。故不存在混合纳什均衡。
	
	\paragraph{5.3}
	
	 考虑如下的双人策略游戏:
	 
	 % Table generated by Excel2LaTeX from sheet 'Sheet1'
	 \begin{table}[htbp]
	 	\centering
	 	\begin{tabular}{|c|c|c|}
	 		\toprule
	 		& 1     & 2 \\
	 		\midrule
	 		1     & [1,1] & [0.5,0.5] \\
	 		\midrule
	 		2     & [0.5,0.5] & [0.5,0.5] \\
	 		\bottomrule
	 	\end{tabular}%
	 	\label{tab:addlabel}%
	 \end{table}%
	 
	显然支配性策略均衡为双方都选择 $1$ 行动,但是恰好有两个纯策略纳什均衡,分别为双方都选择 $1$ 行动或双方都选择 $2$ 行动。
	
	\paragraph{5.4} 
	
	不难发现 $U_1=U_2$,考虑对于矩阵 $V(1,1)=0$ 而其他位置都为 $0$ 的情形。
	
	不难发现此时假设存在纯策略纳什均衡,则将 $a1$ 和 $a2$ 任意一个放大任意正数倍可以发现得到更大 $U_1,U_2$,说明原先的策略不满足纳什均衡,假设不成立。
	
	故不存在纯策略纳什均衡。
	
	\paragraph{5.5}
	
	由于操作集合是有限的,令 $S_a=\{b|a\succeq b\}$,$u(a)=|S_a|$ 即可。
	
	\paragraph{5.6}
	
	考虑如下的二人策略游戏:
	\begin{table}[htbp]
		\centering
		\begin{tabular}{|c|c|c|}
			\toprule
			& 1     & 2 \\
			\midrule
			1     & $[A_{11},B_{11}]$ & $[A_{12},B_{12}]$ \\
			\midrule
			2     & $[A_{21},B_{21}]$ & $[A_{22},B_{22}]$ \\
			\bottomrule
		\end{tabular}%
		
		\label{tab:addlabel}%
	\end{table}%
	
	令 $A=\begin{bmatrix}
		A_{11}& A_{12}\\A_{21}&A_{22}
	\end{bmatrix},B=\begin{bmatrix}
	B_{11}&B_{12}\\B_{21}&B_{22}
\end{bmatrix}$
	
	假设存在随机策略纳什均衡 $a=[a_1,a_2]^T,b=[b_1,b_2]^T$,则 
	$$
	U_1(a,b)=a^TAb,
	U_2(a,b)=a^TBb
	$$
	考虑其成为纳什均衡的必要条件为$$
	\begin{aligned}
		b_1(A_{11}-A_{21})+b_2(A_{12}-A_{22})=0\\
		a_1(B_{11}-B_{12})+a_2(B_{21}-B_{22})=0
	\end{aligned}
	$$
	
	加上概率限制 $a_1+a_2=b_1+b_2=1$,有多组解当且仅当 $A$ 满足 $A_{11}=A_{21},A_{12}=A_{22}$ 或 $B$ 满足 $B_{11}=B_{12},B_{21}=B_{22}$。
	
	$A,B$ 都不满足限制时,存在唯一策略满足纳什均衡条件,不符合题意。
	
	$A,B$ 都满足限制时,任意策略都满足纳什均衡条件,不符合题意。
	
	 $A,B$ 其中一个满足限制时,假设 $A$ 满足限制,则玩家 1 为使纳什均衡成立有唯一可行的策略,此时玩家 2 的任意策略都满足纳什均衡条件,不符合题意。
	 
	 综上,不可能恰好存在 $2$ 组随机策略纳什均衡。
	
\end{document}