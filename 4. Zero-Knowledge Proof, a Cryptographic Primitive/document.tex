\documentclass[12pt, a4paper, oneside]{ctexart}
\usepackage{amsmath, amsthm, amssymb, bm, color, framed, graphicx, hyperref, mathrsfs}
\usepackage{listings}
\usepackage{ctex}

\usepackage{graphicx}
\usepackage{float}
\usepackage{subfigure}

\usepackage{palatino}
\usepackage{tikz}
\usetikzlibrary{shapes.geometric, arrows, graphs, positioning, quotes}

\usepackage{xcolor}
\usepackage{xcolor}
\usepackage{color}
\definecolor{dkgreen}{rgb}{0,0.6,0}
\definecolor{gray}{rgb}{0.5,0.5,0.5}
\definecolor{mauve}{rgb}{0.58,0,0.82}

\usepackage{multirow}
\usepackage{booktabs}

\title{\textbf{TCS作业四:Zero-Knowledge Proof, a Cryptographic Primitive}}
\author{计试2101仲星焱}
\date{\today}

\begin{document}
	\textbf{TCS作业四:Zero-Knowledge Proof, a Cryptographic Primitive}
	
	计试2101仲星焱
	
	\today
	
	\paragraph{4.1} 任取 $c>1,0<\epsilon<1$,令 $p(n) = c^{(n^{\epsilon -1})}$ 即可。
	
	\paragraph{4.2} 考虑题目中的交互式证明系统 $(P,V)$,为了证明其 completeness 与一般交互式证明系统的 completeness 等价,我们考虑对于参数 $n$ ,重复执行证明 $n$ 次,一个 $P$ 被拒绝当且仅当在所有轮次当中都被拒绝,因此一个正确的 $P$ 被拒绝的概率是 ${1}/{3^n}$,是 negligible ,因此一个正确的 $S$ 被大概率接受。此系统的 completeness 不弱于一般的系统
	
	反过来,对于一般的 $(P,V)$ 只需要在接受的时候有 $1/3$ 概率变为拒绝,也可以发现一般系统的 completeness 不弱于此系统。
	
	综上,两种系统的 completeness 等价。
	
	\paragraph{4.3}
	
	1. completeness:正确的 Prover 会被以 $1$ 的概率接受,故此系统 complete
	
	2. soundness:考虑一个 cheating Prover 的策略,显然是每次随机选一个图生成一个同构,然后赌 Verifier 随机选择的图和自己一样,在 $n$ 轮检测当中,全部通过的概率是 $1/2^n$,显然为 negligible,会被以高概率拒绝,故此系统 sound
	
	3. zero-knowledge:考虑 Prover 选取所有同构图的概率是相等的,都是 $1/(n!-2-i)$,其中 $i$ 表示证明已经进行过的轮数(为了保证 $0$ 知识,显然不能生成两个原图)。在获取 input 内容,即两个原图之后,所有轮次生成的 $G''$ 的分布已经确定,返回的映射分布也已经确定,此即 message 的分布,与 Verifier 是否是 cheater 无关,而与其关联的 coin flips 显然也能够通过每轮进行的询问确定分布。故此系统中 Verifier 的 view 的分布可被模拟器完成,所以此系统的 PZK 的。
	
	\paragraph{4.4} 
	
	考虑这样一个系统,P 需要向 V 证明自己手里有一个秘密 $s$ 位二进制串 $\theta$,V 随机生成一个二进制串 $x$ 给 P,P 返回 $x\oplus \theta$ 供 V 解密。
	
	不难发现返回的 $x\oplus \theta$ 在所有 $s$ 位二进制串中均匀分布,此即message 的分布,容易验证这个系统是 $pseudo-ZK$,但是由于 simulator 不知道 $\theta$ 的真实值,故对于每个分布无法推知对应的 coin flips 的状态,反过来也一样,对于特定的 coin flips,无法推知得到的 $x\oplus \theta$ 的分布,故不可能是 PKZ 的。
	
	\paragraph{4.5} 
	
	V 的视野中包含以下内容:$g,h,p,q=2p+1$,每一次收到的message包含 $j$,根据 V 做出的答复还可能收到 $f$ 或 $f'$。
	
	首先容易注意到 $f,f'$ 在分布上存在细微差别,$f$ 为 $\{1,2,\dots,q-1\}$ 当中的均匀分布,而 $f'$ 为 $\{1+e\mod q,2+e\mod q,\dots, q-1+e\mod q\}$,由于我们不知道 $e$ 的具体数值,实际上 $f'$ 的分布无法知道。
	
	因此,我们如此设计 $\sigma$,首先仿照 V 的随机方式生成 $0,1$,决定我们如何生成询问和回答的 message。假设生成 $0$,则按照均匀分布生成 $f$,计算 $j\equiv g^f\pmod p$,
	否则按照 $\{0,1,2,\dots,q-1\}$ 中的均匀分布生成 $f'$,
	计算 $j\equiv g^{f'}h^{-1} \pmod p$。
	其中 $h^{-1}$ 是$\mod p$ 意义下的乘法逆元,可以用 $O(\log p)$ 时间(假设乘法是 $O(1)$) 或者 $O(\log p\log\log p)$(数据过大,乘法不认为是 $O(1)$) 时间预处理。
	
	设$S=\{X|X \text{中有对}Prover \text{的} 1 \text{询问}\}$对于 distinguisher 假设 $D(X)=1$ 当且仅当 $X\in S$,即询问了 $f'$,此时 $|Pr(D(X)=1)-Pr(D(Y)=1)|$ 达到最大,假设进行 $n$ 轮,单次询问中 V 进行 $0$ 询问的概率是 $z$,进行 $1$ 询问的概率是 $w$。有 $$\begin{aligned}
		|Pr(D(X)=1)-Pr(D(Y)=1)|&\leq \sum_{s\in S}|Pr(X=s)-Pr(Y=s)|\\
		&=\sum_{i=0}^nw^iz^{n-i}{n\choose  i}\frac{1}{(2p)^{n-i}}(\frac{1}{(2p)^i}-\frac{1}{(2p+1)^i})\\
		&=\frac{1}{(2p)^n}-(\frac{z}{2p}+\frac{w}{2p+1})^n\\
		&<\frac{1}{(2p)^n}
	\end{aligned}$$
	
	显然 negligible,进而是 \textit{computationally indistinguishable},进而该证明是 CZK的。
	
	话说这不是直接证明是SZK了吗。
		
	\paragraph{4.6}
	
	直接令 $P$ 在最后一轮以 negligible 概率无视询问直接泄露原映射即可。
	
	\paragraph{4.7}
	
	等价,考虑如下证明。
	
	先证明 def 10下可区分,def 8下必然可区分,令 $U=\{u|Pr(A(X)=u)>Pr(A(Y)=u)\}$,则 $$\sum_v|Pr(A(X)=v)-Pr(A(Y)=v)|=2\sum_{v\in U}Pr(A(X)=v)-Pr(A(Y)=v)$$ 非 negligible。
	
	于是我们定义一个 distinguisher  $D,D(X)=[X\in U]$,则对于这个 distinguisher,不难发现 $|Pr(D(X)=1)-Pr(D(Y)=1)|=2\sum_{v\in U}Pr(A(x)=v)-Pr(A(Y)=v)$ 非negligible,但是 def 8 中已经包含了所有 distinguisher,也有这个distinguisher,与假设矛盾,
	
	再证明 def 8 下可区分,def 10下必然可区分,不难发现对于任意 distinguisher ,由加法交换律和绝对值的性质可以知道 $\forall D, |Pr(D(X)=1)-Pr(D(Y)=1)|\leq \sum_v|Pr(A(X)=v)-Pr(A(Y)=v)|$,由于def8下可区分,前者非negligible,后者显然非negligible,进而def10下可区分。
	
	由此,def 8 和 def 10 完全等价。
	
	\paragraph{4.8}
	(a) 不成立,没有保证随机变量独立,一个显然的的反例就是如果 $x,y$ 同分布且 $x$ 与 $\bar{x}$ 同分布,其中 $\bar{x}$ 是 $x$ 的反码串,而 $u = x+1$ ,$v = \bar{y} + 1$ 此处 $+1$ 相当于将 $x,y$ 视为 $s$ 位二进制数进行无符号加法。
	
	不难验证 $x$ 与 $\bar{y}$ 也是同分布的,但是 $(x,u),(y,v)$ 就显然存在较大差异,下面给出一个具体的反例:
	
	$$
	\begin{aligned}
	P(x=0^s)=P(x=1^s)=1/2,\textit{otherwise } P(x=t)=0\\
	P(y=0^s)=P(y=1^s)=1/2,\textit{otherwise } P(y=t)=0\\
	P(u=0^{s-1}1)=P(u=0^s)=1/2,\textit{otherwise } P(u=t) = 0\\
	P(v=0^{s-1}1)=P(v=0^s)=1/2,\textit{otherwise } P(v=t) = 0\\
	\end{aligned}
	$$
	$$
	x=^c y, u=^c v
	$$
	$$
	\begin{aligned}
	P(x=0^s,u=0^{s-1}1)=P(x=1^s,u=0^s)=1/2\\
	P(y=0^s,v=0^s)=P(y=1^s,v=0^{s-1}1)=1/2\\
	\end{aligned}
	$$
	$$
	(x,u)\neq^c (y,v)
	$$
	
	(b) 成立。
	
	不加证明地,我们给出断言,有限个 negligible 的函数/变量之和仍然是 negligible。考虑极限的求和规则显然。
	
	要求考虑所有 distingguisher, 即考虑所有本质不同的取值即可
	
	$$
	\begin{aligned}
	\forall t \in \{0,1\}^s,|Pr(x=t)-Pr(x=t)| &= |\sum_{t'}Pr(x=t,u=t')-\sum_{t'}Pr(y=t,v=t')|\\&\leq \sum_{t'}|Pr(x=t,u=t')-Pr(y=t,v=t')|
	\end{aligned}
	$$
	
	而 $|Pr(x=t,u=t')-Pr(y=t,v=t')|$ 由给出条件知是 negligible的。	
	
	故 $\forall t, |Pr(x=t)-Pr(y=t)|$ 是negligible,$x=^c y$,同理 $u=^c v$。
	
\end{document}