\documentclass[12pt, a4paper, oneside]{ctexart}
\usepackage{amsmath, amsthm, amssymb, bm, color, framed, graphicx, hyperref, mathrsfs}
\usepackage{listings}
\usepackage{ctex}

\usepackage{graphicx}
\usepackage{float}
\usepackage{subfigure}

\usepackage{palatino}
\usepackage{tikz}
\usetikzlibrary{shapes.geometric, arrows, graphs, positioning, quotes}

\usepackage{xcolor}
\usepackage{xcolor}
\usepackage{color}
\definecolor{dkgreen}{rgb}{0,0.6,0}
\definecolor{gray}{rgb}{0.5,0.5,0.5}
\definecolor{mauve}{rgb}{0.58,0,0.82}
\lstset{
	frame=tb,
	language = Matlab,
	aboveskip = 3mm,
	belowskip = 3mm,
	showstringspaces = false,
	backgroundcolor = \color{white}
	columns = flexible,
	basicstyle=\ttfamily\small,
	numbers = left,
	numberstyle = \small \color{gray},
	keywordstyle = \color{blue},
	commentstyle = \color{dkgreen},
	stringstyle = \color{mauve},
	breaklines = true,
	breakatwhitespace = true,
	escapeinside = ``,
	frame = single,
	extendedchars = false,
	xleftmargin = 2em,
	xrightmargin = 2em,
	aboveskip = 1em,
	tabsize = 4
}

\usepackage{multirow}
\usepackage{booktabs}

\title{\textbf{TCS作业一:Turing Machine and Halting}}
\author{计试2101仲星焱}
\date{\today}

\begin{document}
	\textbf{TCS作业一:Turing Machine and Halting}
	
	计试2101仲星焱
	
	\today
	\paragraph{1.1}
	一个比较显然的想法是以一个 $*$ 隔开输入和输出的部分,在输出部分实现二进制的 $+1$ 功能。
	以下假设图灵机读写头初始位于输入串任意位置,输出在输入左侧空一格的位置。
	
	算法具体流程为,每次找到输入串当前存活的最右字符,删去之后对输出进行 $+1$,具体可以参考下表:
	% Table generated by Excel2LaTeX from sheet 'Sheet1'
	\begin{table}[htbp]
		\centering
		\caption{Problem 1.1 图灵机构造}
		\begin{tabular}{|c|c|c|c|c|c|}
			\toprule
			含义    & 状态    & 读到    & S     & W     & M \\
			\midrule
			\multirow{2}[4]{*}{初始状态;扫描输入,向右直到遇到结尾空位置} & \multirow{2}[4]{*}{q0} & 1/0   & q0    & 1/0   & +1 \\
			\cmidrule{3-6}          &       & *     & q1    & *     & -1 \\
			\midrule
			\multirow{2}[4]{*}{处理输入,读到*位置并刚返回,置当前位为*} & \multirow{2}[4]{*}{q1} & 1/0   & q2    & *     & -1 \\
			\cmidrule{3-6}          &       & *     & end   & /     & / \\
			\midrule
			\multirow{2}[4]{*}{扫描输入,前往计数} & \multirow{2}[4]{*}{q2} & 1/0   & q2    & 1/0   & -1 \\
			\cmidrule{3-6}          &       & *     & q3    & *     & -1 \\
			\midrule
			\multirow{2}[4]{*}{执行计数,找连续进位到达的最高位} & \multirow{2}[4]{*}{q3} & 1     & q3    & 0     & -1 \\
			\cmidrule{3-6}          &       & 0/*   & q4    & 1     & +1 \\
			\midrule
			\multirow{3}[6]{*}{计数器更新完毕,前往扫描输入} & \multirow{3}[6]{*}{q4} & 1     & illegal & /     & / \\
			\cmidrule{3-6}          &       & 0     & q4    & 0     & +1 \\
			\cmidrule{3-6}          &       & *     & q0    & *     & +1 \\
			\midrule
			结束状态,直接停机 & end   & /     & /     & /     & / \\
			\midrule
			非法结束状态,磁带状态异常,直接停机 & illegal & /     & /     & /     & / \\
			\bottomrule
		\end{tabular}%
		\label{tab:addlabel}%
	\end{table}%
	
	\paragraph{1.2}
	由于没有详细说明图灵机读写移动和状态转移的顺序,本例中假设一般图灵机的执行顺序为:读状态、写、按照读取的数据进行移动、按照读取的数据进行状态转移。
	
	此解法下,转化后的二进制图灵机状态数是等价三进制图灵机的 $15$ 倍。
	
	输入转化方式即为题设方式,二进制图灵机两位解码三进制图灵机的一位,具体为 $0\rightarrow00,1\rightarrow01,2\rightarrow10,*\rightarrow**$,接下来用 $hi(sign)$ 表示这些编码的高位,用 $lo(sign)$ 表示编码的低位。
	
	对于三进制图灵机 $T=(\Gamma_T,S_T,W_T,M_T)$,构造二进制图灵机 $S=(\Gamma_S,S_S,W_S,M_S)$,其中 $\Gamma_S=\Gamma_T\times\{\#,r_0,r_1,r_*,0++,1++,*++,0--,1--,*--,+,-,0,1,*\}$,共 $15$ 个附加符号,接下来解释这些符号的含义。
	
	$\#$ 表示当前位于某个编码的高位,读取高位之后前进到低位,设当前状态为 $(x,\#)$,有$S_S(x,\#,cur)=(x,r_{cur}),W_S(x,\#,cur)=cur,M_S(x,\#,cur)=+1$。其中 $cur$ 表示当前磁头读取到的数据,即 $0,1$ 或 $*$。
	
	$r_0,r_1,r_*$ 表示刚才读取了某个编码的高位,目前位于马上读取的低位,之后回到高位进行写,低位的写移动和状态转移操作被压缩在剩余的附加符号中,用 $now$ 表示$r$ 附带的数据,$W_S(x,r_{now},cur)=lo(W_T(x,<now,cur>)),M_S(x,r_{now},cur)=-1$,$S_S(x,r_{now},cur)=(S_T(x,<now,cur>), extra), extra$ 具体的取值根据 $W_T,M_T$ 来确定,详细描述为 $extra$ 开头符号为 $hi(W_T(x,<now,cur))$,如果 $M_T(x,<now,cur>)\neq 0$, 则携带两个对应的符号,表示两次位移。
	
	$0++,1++,*++,0--,1--,*--$ 表示当前位于某位的高位,此时磁针探头需要写的数据为符号中数据,在此之后需要向左/右移动进行下一次操作,开始下一次操作时,图灵机 $T$ 对应的状态即为此时 $S$ 携带的状态,$S_S(x,now++,cur)=S_S(x,+),W_S(x,now++,cur)=now, M_S(x,now++,cur)=+1,$ 此处 $--$同理。
	
	$+,-$,表示开始下一次操作之前还需要向指定方向再进行一次移动。 $S_S(x,+,cur)=(x,\#),W_S(x,+,cur)=cur,M_S(x,+,cur)=+1,$ 此处减号同理。
	
	$0,1,*$ 表示填完该位之后直接以下个状态运行,$S_S(x,now,cur)=(x,\#)$,$W_S(x,now,cur)=now,M_S(x,now,cur) = 0$。
	
	综上,即为三进制-二进制图灵机转化策略,由此可以推广说明任意字符集的图灵机计算能力等价。
	
	\paragraph{1.3}
	首先,利用数轴上的整数对 \textit{two-way tape} 的所有单元进行编号,即编号为 $\dots,-3,-2,-1,0,1,2,3\dots$,将 \textit{one-way tape} 每两个单元与 \textit{two-way tape} 每一个单元进行一一对应,从左到右编号为 $0,1,-1,2,-2,3,-3,\dots$。
	
	对于 \textit{two-way tape} 的输入,映射到对应的 \textit{one-way tape} 的第二单元中,另外的第一单元做如下处理, $0$的第一单元置 $0$,正数的第一单元置 $1$,负数的第一单元置 $*$。
	
	于是每次需要对磁头进行移动时,只需要向前移动一单元即可知道实际的移动方向,只需要对 $S$ 中的有限个状态构造 $\Gamma_T=\Gamma_S\times\{-1,0,+1,\}\times\{IO,MV\}$,其中 $IO,MV$ 表示当前状态应该处理磁头为输入/输出信息还是移动信息,前面的数字表示当前需要执行移动的方向,同样为有限个状态。最后,按照上述描述构造对应的移动和输入输出即可。
	
	反过来,对于任意 \textit{one way} 图灵机,显然可以直接用 \textit{two-way} 图灵机进行模拟。
	
	综上,对于 \textit{one-way} 图灵机 $T$,总是存在一个 \textit{two-way} 图灵机 $S$ 与其等价。
	\paragraph{1.4}
	由题意,考虑证明必要性,若 $L$ 是递归可枚举的,根据定义,$\exists T,\forall x\in L, T$ 停机并且输出 $1$,否则停机情况下输出 $0$,采用通用图灵机 $U$ 对 $T$ 进行模拟,若 $T$ 停机,$U$ 读取其输出,若输出为 $1$,$U$ 停机,否则 $U$ 自动进入无穷循环即可。
	
	考虑证明充分性,若 $\exists T,\forall x\in L, T $停机$, \forall x\notin L, T$ 不停机,则采用通用图灵机 $U$ 对 $T$ 进行模拟,$T$ 停机时 $U$ 输出 $1$ 即可,根据定义,$L$ 是递归可枚举的。
	
	综上,原命题成立,$L$ 是递归可枚举的当且仅当存在图灵机 $T$ 当且仅当 $x\in L$ 时停机。
	\paragraph{1.5}
	命题不成立,证明如下:
	
	假设命题成立。构造平凡图灵机 $T$,其接受输入 $(t,x)$ 并模拟图灵机 $t$ 在输入 $x$ 上的运行,显然 $T$ 停机当且仅当 $t$ 在 $x$ 上停机。
	
	由命题成立知道 $\exists T'$,在所有输入 $(t,x)$ 上停机当且仅当 $t$ 在 $x$ 上不停机。
	
	采用通用图灵机 $U$ 对 $T$ 和 $T'$ 进行同步模拟,则我们得到一个停机问题的判定算法,与已知矛盾。
	
	故命题不成立。
	
	\paragraph{1.6}
	该语言不递归,证明如下:
	
	假设该语言递归,则 $\exists S, f_S(T,x,y) = [f_T(x)=f_T(y)]$,其中$[]$ 表示对其中命题判断真假,真命题则返回 $1$,假命题则返回 $0$,根据定义,上述 $S$ 对于任意输入必停机,且 $(T,x,y)$ 作为输入的时候,输出为 $f_S(T,x,y)$ 的值。
	
	现在,利用一个通用图灵机 $U$, 然后找到一个平凡图灵机 $T$ ,使得 $T$ 直接停机并输出 $1$,使用 $U$ 模拟 $T$, 并且 $U$ 停机并输出 $1$ 当且仅当 $T$ 停机并输出 $1$。有 $f_U(T,x) = 1$。
	
	现在,对于任意图灵机 $T'$ 和任意输入 $x'$,有 $f_S(U,(T,x),(T',x')) = [f_U(T,x)=f_U(T',x')] = f_U(T',x')$,即可利用 $S$ 计算 $T'$ 在 $x'$ 上是否停机,与停机问题结论互斥,故假设不成立。
	
	综上,该语言不递归。
	\paragraph{1.7}
	
	(a)是,考虑如下证明:
	
	设该语言为 $L_1$,使用通用图灵机 $U$ 对 $T,S$ 进行输入为 $x$ 的同步模拟,在$T,S$出现停机时 $U$ 输出结果并停机,对于 $\forall (T,S,x)\in L_1$,$T$ 在 $x$ 上必然停机,故 $U$ 在 $(T,S,x)$ 上必然停机。另取一个通用图灵机 $D$ 模拟 $U$ 即可证明语言 $L_1$ 是递归可枚举的。
	
	(b)不是,考虑反证:
	
	设该语言为 $L_2$,由于其是递归可枚举的,$\exists B$,使得 $\forall (T,S,x)\in L_2$,$B$ 都停机并做出输出。另外设 $(a)$ 中所得图灵机为 $A$。
	
	构造平凡图灵机 $S$,使得 $S$ 对任意输入都不停机。对于任意图灵机 $T$ 和输入 $x$,要么 $T$ 比 $S$ 快,要么 $S$ 快于或等于 $T$, 两种情况分别对应 $T$ 在 $x$ 上停机或不停机。
	
	对于停机的情况,在 $A$ 上运行 $(T,S,x)\in L_1$ 必然让 $A$ 停机。
	
	对于不停机的情况,在 $B$ 上运行 $(S,T,x)\in L_2$ 必然让 $B$ 停机。
	
	采用通用图灵机 $U$ ,对 $A$ 运行输入为 $(T,S,x)$ 和对 $B$ 运行输入为 $(S,T,x)$ 进行同步模拟,直到 $A$ 和 $B$ 其中一个停机(由前述可知两者必然有一个停机),停机的一方即暗示了 $T$ 在 $x$ 上是否停机,由此我们找到一个停机问题的判定算法,与已知停机问题无图灵判定算法矛盾,故假设不成立。
	
	综上,该语言不是递归可枚举的。
	
\end{document}