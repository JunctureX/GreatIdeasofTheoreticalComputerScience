\documentclass[12pt, a4paper, oneside]{ctexart}
\usepackage{amsmath, amsthm, amssymb, bm, color, framed, graphicx, hyperref, mathrsfs}
\usepackage{listings}
\usepackage{ctex}

\usepackage{graphicx}
\usepackage{float}
\usepackage{subfigure}

\usepackage{palatino}
\usepackage{tikz}
\usetikzlibrary{shapes.geometric, arrows, graphs, positioning, quotes}

\usepackage{xcolor}
\usepackage{xcolor}
\usepackage{color}
\definecolor{dkgreen}{rgb}{0,0.6,0}
\definecolor{gray}{rgb}{0.5,0.5,0.5}
\definecolor{mauve}{rgb}{0.58,0,0.82}

\usepackage{multirow}
\usepackage{booktabs}

\title{\textbf{TCS作业七:Condition Number and Ill-Conditioned Matrix}}
\author{计试2101仲星焱}
\date{\today}

\begin{document}
	\textbf{TCS作业七:Condition Number and Ill-Conditioned Matrix}
	
	计试2101仲星焱
	
	\today
	
	\paragraph{7.1}
	
	不是一般性地,设 $|a_1|>|a_2|>\cdots>|a_n|$。
	
	不难发现 $\forall v \in \mathbb R^n,Av$ 能够的取值只有 $\textit{span}\{a_1,a_2,\dots,a_n\}$,由于其中向量相互正交,在 $|v|=1$ 的情况下,设 $v=[v_1,v_2,\dots,v_n]^T$,值 $|Av|=\sqrt{\sum\limits_{i=1}^n|a_i|^2v_i^2}\leq |a_1|$,该小于等于号由柯西不等式易得,取等当且仅当 $v_1=1,v_i=0\forall i\neq 1$。
	
	后续的奇异向量由于须与之前的正交,不难发现其能够的取值在 $\textit{span}\{a_{i+1},\dots,a_n\}$ 当中,以此类推,可以发现 $A$ 的奇异值从大到小即为所有 $|a_i|$ 的从大到小排列。
	
	\paragraph{7.2}
	
	只需要注意到 $\lambda x = Ax\Rightarrow |\lambda|=|Ax|/|x|$ 而 $\sigma_1=\max\limits_{|x|=1,x\in \mathbb R^n}|Ax|$,显然有 $\sigma_1\geq |\lambda|$ 即对于实对称矩阵,最大的奇异值不小于任何特征值的绝对值。
		
	考虑到 $\frac{1}{\lambda_n}$ 和 $\frac{1}{\sigma_n}$ 分别是 $A^{-1}$ 绝对值最大的特征值和奇异值,有 $\frac{1}{\lambda_n} \leq \frac{1}{\sigma_n}$。
	
	由此 $\frac{|\lambda_1|}{|\lambda_n|}\leq \frac{\sigma_1}{\sigma_n}=||A||\cdot||A^{-1}||=\textbf{cond}(A)$。
	
	\paragraph{7.3} 由于$\forall v \in \mathbb R^n, v^TA^TAv=|Av|^2\geq 0$,而取等号当且仅当 $Av = 0$,而由于 $A$ 非奇异,故当且仅当 $v=0$ 可以取等号,由此得到 $A^TA$ 是正定矩阵。
	
	\paragraph{7.4} 
	不失 一般性地,设 $|\lambda_1|>|\lambda_2|>\cdots>|\lambda_n| $
	
	考虑 $A$ 的单位正交分解, $A=P^TDP$,其中 $P$ 是单位正交矩阵, $D=\textit{diag}(\lambda_1,\lambda_2,\dots,\lambda_n)$。 
	
	由此,考虑其最大奇异值 $\max\limits_{|v|=1}|Av|=\max\limits_{|v|=1}|P^TDPv|$,而 $P$ 和 $P^T$ 都是单位正交矩阵,取 $w=Pv$ 显然 $|w|=1$,故 $\sigma_1=\max\limits_{|w|=1}|Dw|\leq |\lambda_1|$,不等号由柯西不等式可得。
	
	同理,以此类推,由于所有特征值互不相同,所有奇异值即为所有特征值的绝对值。
	
\end{document}